\subsection{Resultados de Pruebas con Diferentes Cargas de Usuarios Virtuales}

A continuación se presentan los resultados obtenidos de la ejecución de las pruebas con distintos usuarios virtuales (VUs), los cuales se resumen en la Tabla \ref{tab:results_vus}. Se evaluaron escenarios con 20, 50, 100, 150, 200 y 300 VUs para analizar la latencia, la robustez y el throughput del sistema bajo distintas cargas.

\begin{table}[H]
\centering
\begin{tabularx}{\textwidth}{|c|X|X|X|X|}
\hline
\textbf{VUs} & \textbf{http\_req\_duration (p95)} & \textbf{http\_req\_failed} & \textbf{Checks de éxito} & \textbf{Throughput (reqs/s)} \\ \hline
20  & 462.78 ms & 0.00\% & 100\% & 13.12 \\ \hline
50  & 452.21 ms & 0.00\% & 100\% & 42.73 \\ \hline
100 & 460.09 ms & 0.00\% & 100\% & 72.28 \\ \hline
150 & 452.52 ms & 0.00\% & 100\% & 102.69 \\ \hline
200 & 454.07 ms & 0.00\% & 100\% & 132.69 \\ \hline
300 & 454.67 ms & 0.00\% & 100\% & 193.08 \\ \hline
\end{tabularx}
\caption{Resumen de métricas principales obtenidas para diferentes cargas de usuarios virtuales.}
\label{tab:results_vus}
\end{table}

\subsubsection{Análisis de Métricas Principales}

Como se observa en la Tabla \ref{tab:results_vus}, todos los escenarios evaluados mantienen la latencia p95 por debajo del umbral de 500 ms, con una tasa de errores nula y 100\% de verificaciones exitosas. Esto evidencia la robustez del servidor bajo cargas progresivamente mayores.

El throughput aumenta proporcionalmente con el número de usuarios virtuales, reflejando la capacidad de manejo de tráfico del servidor.

\subsubsection{Estadísticas de Latencia}

Los tiempos de respuesta registrados muestran estabilidad adecuada bajo todas las cargas evaluadas:

\begin{itemize}
    \item \textbf{20 VUs:} avg = 128.39 ms, med = 2.56 ms, max = 513.01 ms
    \item \textbf{50 VUs:} avg = 128.52 ms, med = 2.27 ms, max = 515.8 ms
    \item \textbf{100 VUs:} avg = 129.77 ms, med = 2.65 ms, max = 513.48 ms
    \item \textbf{150 VUs:} avg = 128.02 ms, med = 2.3 ms, max = 515.9 ms
    \item \textbf{200 VUs:} avg = 127.83 ms, med = 2.84 ms, max = 513.22 ms
    \item \textbf{300 VUs:} avg = 127.25 ms, med = 3 ms, max = 513.21 ms
\end{itemize}

El sistema logró procesar un total de peticiones que varía según la carga, con un rendimiento creciente de throughput (reqs/s) proporcional al número de VUs, sin registrar fallos en ninguna de las pruebas.

\subsubsection{Interpretación General}

Como se puede observar en la Tabla \ref{tab:results_vus}:

\begin{itemize}
    \item La latencia p95 se mantiene estable y por debajo del umbral de 500 ms, incluso al aumentar la carga a 300 VUs.
    \item La tasa de errores HTTP permanece en 0\%, evidenciando la robustez de la API.
    \item El throughput aumenta de manera lineal conforme se incrementa el número de usuarios virtuales, mostrando escalabilidad del sistema.
    \item Todos los checks definidos en k6 se completan satisfactoriamente, asegurando la correcta funcionalidad de los endpoints bajo cargas crecientes.
\end{itemize}

Estos resultados permiten concluir que la API es capaz de manejar cargas crecientes de usuarios virtuales manteniendo latencias bajas y cero errores, demostrando estabilidad y escalabilidad.

