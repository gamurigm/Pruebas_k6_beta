\section{Introducción}

Este informe presenta el desarrollo y los resultados obtenidos en el laboratorio de pruebas de carga y rendimiento de servicios API REST. El propósito fundamental es evaluar la capacidad de respuesta y estabilidad de una interfaz programática ante escenarios de uso intensivo mediante la simulación de usuarios concurrentes.

El aseguramiento de la calidad en el software moderno requiere no solo verificar la funcionalidad, sino también garantizar que el sistema pueda operar bajo carga sin degradar la experiencia del usuario final \cite{sahli2023load}. Para este propósito, la utilización de herramientas de orquestación de carga como k6 permite realizar pruebas de rendimiento de manera programática, facilitando la identificación de cuellos de botella en arquitecturas de backend distribuidas \cite{vukovic2020performance}.

En las siguientes secciones se detalla el proceso de configuración del ambiente, la creación de los scripts de prueba y el análisis exhaustivo de las métricas obtenidas, permitiendo identificar posibles cuellos de botella y áreas de mejora en la infraestructura del servidor.
