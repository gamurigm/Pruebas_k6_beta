\section{Marco Teórico}

\subsection{Node.js y Express}
Node.js es un entorno de ejecución de JavaScript orientado a eventos, diseñado para construir aplicaciones de red escalables. Express es el framework web más popular para Node.js, facilitando la creación de APIs REST mediante el manejo eficiente de rutas y middleware.

\subsection{k6 y las Pruebas de Carga}
k6 es una herramienta de pruebas de rendimiento de código abierto, moderna y centrada en el flujo de trabajo del desarrollador. En el contexto de arquitecturas modernas, las pruebas de carga permiten mitigar riesgos asociados a la escalabilidad y al consumo de recursos en entornos de producción saturados \cite{bezemer2019performance}. Esta herramienta permite escribir scripts en JavaScript, optimizando la simulación de usuarios virtuales (VUs) con un consumo eficiente de memoria.

\subsection{Métricas de Rendimiento}
Las métricas de rendimiento son indicadores cuantitativos del comportamiento de un sistema bajo carga y son fundamentales para evaluar la calidad del servicio \cite{vukovic2020performance}. Entre las más relevantes se encuentran:

\begin{itemize}
    \item \textbf{Tiempo de respuesta (latencia):} Mide cuánto tarda el sistema en responder a una solicitud. Los percentiles (por ejemplo p95) son ampliamente usados para entender la variabilidad y comportamiento de cola de las respuestas \cite{chucksacademyPerformance}.
    \item \textbf{Throughput:} Representa la cantidad de solicitudes exitosas procesadas por segundo (reqs/s), lo que indica la capacidad de manejo de tráfico del sistema \cite{turn0search1}.
    \item \textbf{Tasa de peticiones fallidas:} Proporción de solicitudes que resultan en errores (como códigos HTTP 4xx o 5xx), indicador de robustez ante carga \cite{turn0search1}.
    \item \textbf{Usuarios Virtuales (VUs):} Número de instancias que simulan usuarios simultáneos, utilizado para modelar carga concurrente en pruebas de rendimiento \cite{turn0search5}.
\end{itemize}

Estas métricas permiten determinar el punto de saturación de un backend y ofrecen una base para comparaciones objetivas entre diferentes configuraciones o versiones de un sistema \cite{turn0search9}.
